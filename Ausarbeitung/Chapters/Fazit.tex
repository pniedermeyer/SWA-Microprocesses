\chapter{Fazit}

Letztendlich erreicht die Microservice Architektur ihr Ziel: Systeme können sehr gut auf viele Server verteilt und damit skaliert werden. Allerdings ist die Architektur nicht in jeder Situation optional, denn für jeden Vorteil, den die Architektur gegenüber der klassischen Monolith Architektur hat, existiert ein Nachteil, der andere Probleme macht. Beispielsweise sind die Daten zwar sehr verteilt, dafür ist es allerdings deutlich schwerer die Daten als Ganzes zu verwalten und Transaktionen sind praktisch nur sehr schwer umsetzbar. Letztendlich bleibt jedoch der Vorteil der Skalierbarkeit bestehen und wenn bei einem großem System die Wahl zwischen den skalierenden Microservices und einem nicht skalierendem Monolithen besteht, dann müssen selbstverständlich die Microservices gewählt werden, da das System sonst schlicht nicht umsetzbar ist. Ein Monolith ist allerdings eine völlig valide und gerechtfertigte Wahl, wenn die Skalierbarkeit eines Microservice-Systems nicht benötigt wird.
