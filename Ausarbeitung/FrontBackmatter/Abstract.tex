%*******************************************************
% Abstract
%*******************************************************
\pdfbookmark[0]{Zusammenfassung}{Zusammenfassung}
\chapter*{Zusammenfassung}
Die vorliegende Arbeit beschäftigt sich mit dem Software Architekturstil „Microservice“. Ziel dieser Arbeit ist es den Architekturstiel Microservice genauer zu betrachten, d.h. es wir geschaut was sich hinter diesem Architekturstiel verbirgt, wie dieser funktioniert bzw. wie dieser Aufgebaut ist und weshalb man diese Architektur verwenden sollte bzw. in welchen Situationen es von Vorteil ist diese Architektur zu verwenden.\newline\newline
Zu Beginn wird eine kurze Einführung geliefert. Diese bietet dem Leser eine Übersicht darüber mit was sich diese Arbeit beschäftigt und worauf der Fokus gelegt wird. Nach der Einführung wird ein überblicke über den Microservice an sich sowie die Funktionsweise gegeben. Anschließend wird die Microservice-Architektur mit anderen Architekturstielen verglichen.\newline In den beiden darauffolgenden Kapiteln wird sowohl auf die Vorteile als auch auf die Herausforderung bei der Nutzung von Microservices geschaut. Das Kapitel Daten beschäftigt sich mit den Problemen und Herausforderungen mit Daten mit denen Entwickler bei der Arbeit mit Microservices konfrontiert werden. Die Fallstudie zeigt ein kleines Beispiel für einen möglichen Microservice wie er in der Geschäftswelt auftreten kann. Das Fazit beschäftigt sich mit einem Rückblick über die Arbeit, sowie einem Ausblick für die zukünftige Nutzung von Microservices.