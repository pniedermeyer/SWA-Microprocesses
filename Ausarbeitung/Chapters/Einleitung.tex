\chapter{Einleitung}

In der modernen Welt nutzen immer mehr Menschen immer komplexere Services, die über das Internet bereitgestellt werden. Firmen wie Google, Netflix und Amazon müssen täglich Millionen von Anfragen bearbeiten. Solche großen Systeme stoßen sehr schnell an die Grenzen eines einzigen Servers, da es nicht möglich ist genügend Rechenleistung und Speicher zu verbauen, um eine reibungslose Nutzung zu ermöglichen. Der klassische, monolithische Ansatz, bei dem ein einziges System auf einem einzigen Server läuft, funktioniert also nicht mehr. Stattdessen müssen verteilte Systeme, die die Leistung von vielen Servern (bis hin zu ganzen Rechenzentren) nutzen können, eingesetzt werden. Diese, verteilte Systeme, sind weitaus skalierbarer als klassische monolithische Systeme.
\paragraph{}
Microservices sind ein Architektur-Pattern, das genutzt werden kann, um die Funktionalität eines Systems in kleine Unter-Systeme aufzuteilen. Mithilfe dieser wird dann das große Gesamtsystem aufgebaut. Die Skalierbarkeit entsteht dadurch, dass jeder Microservice, bei Notwendigkeit, auf einem eigenen Server laufen kann. Im Folgenden wird genauer auf das grundlegende Konzept von Microservices eingegangen, sowie auf die Vorteile und Herausforderungen, die eine Microservice Architektur mit sich bringt. Letztendlich wird in einem Minimal-Fallbeispiel die Architektur gezeigt.
